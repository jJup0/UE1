\documentclass[a4paper,%
11pt,%
DIV=12,
headsepline,%
headings=normal,
]{scrartcl}

\usepackage[utf8]{inputenc}
\usepackage[T1]{fontenc}
\usepackage[automark]{scrlayer-scrpage}
\usepackage{graphicx}
\usepackage{lmodern} 
\usepackage{url}
\usepackage{amsmath}
\usepackage{amssymb}
\usepackage{booktabs}
\usepackage{listings}

\lstset{
  basicstyle=\ttfamily\footnotesize,
  frame=single
}

\newcounter{curex}
\setcounter{curex}{0}
\newcommand{\exercise}[1]{\section*{Exercise #1}\setcounter{curex}{#1}}
\newcommand{\answer}[1]{\subsection*{Answer \arabic{curex}.#1}}

\begin{document}

\noindent
\vspace*{1ex}
\begin{minipage}[t]{.45\linewidth}
\strut\vspace*{-\baselineskip}\newline
\includegraphics[height=.9cm]{./figs/Inf-Logo_black_en-eps-converted-to.pdf}
\includegraphics[height=.9cm]{./figs/par-logo}
\end{minipage}
\hfill
\begin{minipage}[t]{.5\linewidth}
\flushright{
Research Group for Parallel Computing\\%
Faculty of Informatics\\%
TU Wien}
\end{minipage}
\vspace*{1ex}

\hrule 

\vspace*{2ex}

\begin{center}
{\LARGE\textbf{Parallel Computing}}\\
{\large{}%
  2022S\\
  Übungsblatt 1\\
}
\end{center}

\hrule 
\vspace*{1ex}

\noindent
1: First name, Last name, Matrikel\\
2: First name, Last name, Matrikel\\
3: First name, Last name, Matrikel

\vspace*{1ex}
\hrule 

\exercise{1}

The source code snippets are just given here in case you would like to reuse them.
You can remove them in your solution if you want.
\begin{lstlisting}
void mv(int m, int n, double M[m][n], double V[n], double W[m])
{
  int i, j;

  for (i=0; i<m; i++) {
    W[i] = 0.0;
    for (j=0; j<n; j++) {
      W[i] += M[i][j]*V[j];
    }
  }
}
\end{lstlisting}

\answer{1}

\answer{2}

\answer{3}

\exercise{2}

\begin{lstlisting}
for (i=0; i<n; i++) {
  int count = 0;
  for (j=0; j<i; j++) {
    if (a[j]<=a[i]) count++;
  }
  j++;
  for (; j<n; j++) {
    if (a[j]<a[i]) count++;
  }
  b[count] = a[i];
}
for (i=0; i<n; i++) a[i] = b[i];
\end{lstlisting}

\answer{1}

\answer{2}

\answer{3}

\answer{4}

\answer{5}



\exercise{3}

\answer{1}

\answer{2}

\begin{tabular}{rrrrrrrrrrrrrrrrrrrr}
  \toprule
 case /  $i$ & 0 & 1 & 2 & 3 & 4 & 5 & 6 & 7 & 8 & 9 & 10 & 11 & 12 & 13 & 14 & 15 & 16 & 17 & 18  \\
  \midrule
\texttt{static} &  \multicolumn{8}{c}{insert thread ids}\\
\texttt{static,2} & \\
\texttt{static,5} & \\
\texttt{static,6} & \\ 
\texttt{dynamic,1} & \\
\texttt{dynamic,2} & \\
  \texttt{guided,3} & \\
  \bottomrule
\end{tabular}\\

\begin{tabular}{rrrrrrr}
  \toprule
case / $t$             & $t[0]$ & $t[1]$ & $t[2]$ & $t[3]$ & $t[4]$ & $t[5]$ \\  
  \midrule
\texttt{static} & \\
\texttt{static,2} & \\
\texttt{static,5} &  \\
\texttt{static,6} &  \\
\texttt{dynamic,1} &  \\
\texttt{dynamic,2} &  \\
\texttt{guided,3} &  \\
    \bottomrule
\end{tabular}

\answer{3}


\exercise{4}

\begin{lstlisting}
i = 0; j = 0; k = 0; 
while (i<n&&j<m) {
   C[k++] = (A[i]<=B[j]) ? A[i++] : B[j++]; 
}
while (i<n) C[k++] = A[i++]; 
while (j<m) C[k++] = B[j++];
\end{lstlisting}

\answer{1}

\answer{2}

\exercise{5}

\answer{1}

\answer{2}

\answer{3}

\exercise{6}

\begin{lstlisting}
void merge_corank(double A[], int n, double B[], int m, double C[])
{
  int t; // number of blocks (threads)
  int i;
  
  int coj[t+1];
  int cok[t+1];
  
  for (i=0; i<t; i++) {
    corank(i*(n+m)/t,A,n,&coj[i],B,m,&cok[i]);
  }
  coj[t] = n;
  cok[t] = m;
  
  for (i=0; i<t; i++) {
    merge(&A[coj[i]],coj[i+1]-coj[i],
	  &B[cok[i]],cok[i+1]-cok[i],
	  &C[i*(n+m)/t]);
  }
}
\end{lstlisting}

\answer{1}

\answer{2}

\answer{3}

\answer{4}

\exercise{7}

\begin{lstlisting}
void merge_divconq(double A[], int n, double B[], int m, double C[])
{
  int i;

  if (n==0) { // task parallelize for large n
    for (i=0; i<m; i++) C[i] = B[i];
  } else if (m==0) { // task parallelize for large m
    for (i=0; i<n; i++) C[i] = A[i];
  } else if (n+m<CUTOFF) {
    merge(A,n,B,m,C); // sequential merge for small problems
  } else {
    int r = n/2;
    int s = rank(A[r],B,m);
    C[r+s] = A[r];
    merge_divconq(A,r,B,s,C);
    merge_divconq(&A[r+1],n-r-1,&B[s],m-s,&C[r+s+1]);
  }
}
\end{lstlisting}

\answer{1}

\answer{2}

\answer{3}

\exercise{8}

\answer{1}

\answer{2}

\answer{3}

\exercise{9}

\answer{1}

\answer{2}

\answer{3}


\end{document}
