\documentclass[a4paper,%
11pt,%
DIV=12,
headsepline,%
headings=normal,
]{scrartcl}

\usepackage[utf8]{inputenc}
\usepackage[T1]{fontenc}
\usepackage[automark]{scrlayer-scrpage}
\usepackage{graphicx}
\usepackage{lmodern} 
\usepackage{url}
\usepackage{amsmath}
\usepackage{amssymb}
\usepackage{booktabs}
\usepackage{listings}

\usepackage{graphicx}
\graphicspath{ {../plots/} }

\lstset{
  basicstyle=\ttfamily\footnotesize,
  frame=single
}

\newcounter{curex}
\setcounter{curex}{0}
\newcommand{\exercise}[1]{\section*{Exercise #1}\setcounter{curex}{#1}}
\newcommand{\answer}[1]{\subsection*{Answer \arabic{curex}.#1}}

\newcommand\plotwidth{0.8\textwidth}
\newcommand\plotheight{0.48\textwidth}

\begin{document}

\noindent
\vspace*{1ex}
\begin{minipage}[t]{.45\linewidth}
\strut\vspace*{-\baselineskip}\newline
\includegraphics[height=.9cm]{./figs/Inf-Logo_black_en-eps-converted-to.pdf}
\includegraphics[height=.9cm]{./figs/par-logo}
\end{minipage}
\hfill
\begin{minipage}[t]{.5\linewidth}
\flushright{
Research Group for Parallel Computing\\%
Faculty of Informatics\\%
TU Wien}
\end{minipage}
\vspace*{1ex}

\hrule 

\vspace*{2ex}

\begin{center}
{\LARGE\textbf{Parallel Computing}}\\
{\large{}%
  2022S\\
  Übungsblatt 1\\
}
\end{center}

\hrule 
\vspace*{1ex}

\noindent
1: Jakob, Roithinger, 52009269\\
2: Elias, Pinter, 12023962\\
3: First name, Last name, Matrikel

\vspace*{1ex}
\hrule 

\exercise{1}

The source code snippets are just given here in case you would like to reuse them.
You can remove them in your solution if you want.
\begin{lstlisting}
void mv(int m, int n, double M[m][n], double V[n], double W[m])
{
  int i, j;

  for (i=0; i<m; i++) {
    W[i] = 0.0;
    for (j=0; j<n; j++) {
      W[i] += M[i][j]*V[j];
    }
  }
}
\end{lstlisting}

\answer{1}
\begin{lstlisting}
void mv(int m, int n, double M[m][n], double V[n], double W[m]){
  par(0<=i<m){
    W[i] = 0.0;
    for(int j = 0; j < n; j++){
    	
      // summing into each cell in result vector 
      W[i] += M[i][j] * V[j];
    }
  }
}
\end{lstlisting}
requires the CREW PRAM model
\newpage
\answer{2}
\begin{lstlisting}
void mv(int m, int n, double M[m][n], double V[n], double W[m]){

  par(0<=i<m, 0<=j<n){	
    // calculating summands in O(1)time
    M[i][j] *= V[j];
  }
  
  par(0<=i<m, 0<=j<n){
    // summing in O(log(n)) time
    int offset = 1;
    int betweenGap = 2;
    while(offset < n){
      if(offset + j * betweenGap >= n){
        noop;
      }
      else{
        M[i][j * betweenGap] += M[i][offset + j * betweenGap]
      }
      offset *= 2;
      betweenGap*=2;
    }
  }
  par(0 <= i < n){
    // copying into result vector in O(1) time
    W[i] = M[i][0];
  }
\end{lstlisting}
requires the CREW PRAM model
\newpage
\answer{3}
\begin{lstlisting}
void mv(int m, int n, double M[m][n], double V[n], double W[m]){
  double C[m][n];
  par(0 <= j < n){
    // copying vector into intermediate matrix in O(1) time
    C[0][j] = V[j];
  }

  par(0<=i<m, 0<=j<n){
    // copying vector in O(log(n)) time
    range = 2;
    offset = 1;
    while(offset < m){
      if(i + offset >= range && i + offset >= m){
        noop;
      }
      else{
        C[i + offset][j] = C[i][j];
        offset *= 2;
        range *= 2;
      }
    }
		
    // calculating summands in O(1)time
    M[i][j] = M[i][j] * C[i][j];

    
    // summing in O(log(m)) time
    int offset = 1;
    int betweenGap = 2;
    while(offset < n){
      if(offset + j * betweenGap >= n){
        noop;
      }
      else{
        M[i][j * betweenGap] += M[i][offset + j * betweenGap]
      }
      offset *= 2;
      betweenGap*=2;
    }
  }
  par(0 < i < n){
    // copying into result vector in O(1) time
    W[i] = M[i][0];
  }
\end{lstlisting}
requires the EREW PRAM model

\exercise{2}

\begin{lstlisting}
for (i=0; i<n; i++) {
  int count = 0;
  for (j=0; j<i; j++) {
    if (a[j]<=a[i]) count++;
  }
  j++;
  for (; j<n; j++) {
    if (a[j]<a[i]) count++;
  }
  b[count] = a[i];
}
for (i=0; i<n; i++) a[i] = b[i];
\end{lstlisting}

\answer{1}
The program sorts a given array of integers in ascending order. A comparison with "$a \leq \ $b" has to be done in some range, if the method is also supposed to work with arrays holding duplicates. If this is not done, all duplicates will end up on the same position, causing ''gaps'' (array position will have standard initialization like 0 or null) in the result. If "$a \leq \ $b" is used in the place where it is in the current code, it will guarantee stability. This means that the order of equal elements in the input will be sustained in the output.

\answer{2}
The asymptotic sequential work of the given algorithm is O(n$^{2}$). The outer loop is executed n times. Within this loop, there are two nested loops which have a total of n-1 (= O(n)) iterations. Outside the outer loop, there is another loop which is executed n times. Every operation in- and outside the loops does have constant work. This concludes to the work of O(n$^{2}$). Here is the code once again, with explanation regarding work in the comments:

\begin{lstlisting}
for (i=0; i<n; i++) {            // canonical loop dependent on n
  int count = 0;                 // constant work
  for (j=0; j<i; j++) {          // canonical loop dependent on i
    if (a[j]<=a[i]) count++;     // constant work
  }
  j++;                           // constant work
  for (; j<n; j++) {             // linear loop dependent on the i-loop and n
    if (a[j]<a[i]) count++;      // constant work
  }
  b[count] = a[i];               // constant work
}
for (i=0; i<n; i++)              // canonical loop dependent on n
  a[i] = b[i];                   // constant work
\end{lstlisting}
\begin{math}n*(1 + n*(1) + 1 + n*(1) + 1) + n*(1) =\end{math}\\
\begin{math}n*(n + 3) + n =\end{math}\\
\begin{math}n^2 + 4n\end{math}\\
\begin{math}n^2 + 4n \in O(n^2)\end{math}\\
This is a very simplified calculation. The j-loop's workload rises as the i-loop's workload shrinks. Generally, within one outer-loop iteration, the sum of the iterations of the first and second j-loop together is n. However, it does not make in difference for the big O notation.

\answer{3}
\begin{lstlisting}
#include <omp.h>
...
a = (int*)malloc(n*sizeof(int));
b = (int*)malloc(n*sizeof(int));
#pragma omp parallel default(none) shared(a, b)
{
 int i;
 #pragma omp for
  for (i=0; i<n; i++) {
   int count = 0;
   for (j=0; j<i; j++) {
     if (a[j]<=a[i]) count++;
   }
   j++;
   for (; j<n; j++) {
     if (a[j]<a[i]) count++;
   }
   b[count] = a[i];
 }
}
// implicit barrier
int i;
#pragma omp parallel for
{
 for (i=0; i<n; i++) a[i] = b[i];
}
\end{lstlisting}

\answer{4}
The outer loop is embarrassingly parallel, because the iterations are independent of each other. Therefore, the speedup is linear (compared to the sequential execution of this algorithm). The runtime will be in:\\
\begin{math}O(\frac{n^2}{p})\end{math}\\

\answer{5}
Relative speedup will be linear as long as $p \leq n$.\\

\begin{math}SRel_{p}(n)=\frac{T_{par}(1,n)}{T_{par}(p,n)} = \frac{n^2}{n^2/p} = p \end{math}\\
The absolute speedup has to be compared with the best known sequantial algorithm for this problem (sorting). Mergesort is on of these, running in \begin{math}O(n \cdot log(n))\end{math}. The absulote speed-up is not linear.\\

\begin{math}S_{p}(n)=\frac{T_{seq}(n)}{T_{par}(p,n)} = \frac{n \cdot log(n)}{n^2/p} = \frac{log(n) \cdot p}{n} \end{math}\\



\exercise{3}
\begin{lstlisting}
# pragma omp parallel for schedule ( runtime )
for (i = 0; i < n ; i++) {
  a[i] = omp_get_thread_num();
  t[omp_get_thread_num()]++;
}
\end{lstlisting}
\answer{1}
a[i] tracks which thread executed iteration \emph{i} in the for loop.
t[i] tracks how many iterations were executed by thread \emph{i}.

\answer{2}


\begin{tabular}{rrrrrrrrrrrrrrrrrrrr}
  \toprule
 case /  $i$ & 0 & 1 & 2 & 3 & 4 & 5 & 6 & 7 & 8 & 9 & 10 & 11 & 12 & 13 & 14 & 15 & 16 & 17 & 18  \\
  \midrule
\texttt{static}     & 0 & 0 & 0 & 1 & 1 & 1 & 2 & 2 & 2 & 3 & 3 & 3 & 4 & 4 & 4 & 5 & 5 & 5 & 0 \\
\texttt{static,2}   & 0 & 0 & 1 & 1 & 2 & 2 & 3 & 3 & 4 & 4 & 5 & 5 & 0 & 0 & 1 & 1 & 2 & 2 & 3 \\
\texttt{static,5}   & 0 & 0 & 0 & 0 & 0 & 1 & 1 & 1 & 1 & 1 & 2 & 2 & 2 & 2 & 2 & 3 & 3 & 3 & 3 \\
\texttt{static,6}   & 0 & 0 & 0 & 0 & 0 & 0 & 1 & 1 & 1 & 1 & 1 & 1 & 2 & 2 & 2 & 2 & 2 & 2 & 3 \\ 
\texttt{dynamic,1}  & 0 & 1 & 2 & 3 & 4 & 5 & 0 & 1 & 2 & 3 & 4 & 5 & 0 & 1 & 2 & 3 & 4 & 5 & 0 \\
\texttt{dynamic,2}  & 0 & 0 & 1 & 1 & 2 & 2 & 3 & 3 & 4 & 4 & 5 & 5 & 0 & 0 & 1 & 1 & 2 & 2 & 3 \\
  \texttt{guided,3} & 0 & 0 & 0 & 1 & 1 & 1 & 2 & 2 & 2 & 3 & 3 & 3 & 4 & 4 & 4 & 5 & 5 & 5 & 0 \\
  \bottomrule
\end{tabular}\\

\begin{tabular}{rrrrrrr}
  \toprule
case / $t$             & $t[0]$ & $t[1]$ & $t[2]$ & $t[3]$ & $t[4]$ & $t[5]$ \\  
  \midrule
\texttt{static}    & 4 & 3 & 3 & 3 & 3 & 3 \\
\texttt{static,2}  & 4 & 4 & 4 & 3 & 2 & 2 \\
\texttt{static,5}  & 5 & 5 & 5 & 4 & 0 & 0 \\
\texttt{static,6}  & 6 & 6 & 6 & 1 & 0 & 0 \\
\texttt{dynamic,1} & 4 & 3 & 3 & 3 & 3 & 3 \\
\texttt{dynamic,2} & 4 & 4 & 4 & 3 & 2 & 2 \\
\texttt{guided,3}  & 4 & 3 & 3 & 3 & 3 & 3 \\
    \bottomrule
\end{tabular}

\answer{3}
False sharing is a possible performance issue:
Multiple threads (on different cores) may want to access t (at different indexes) at close points in time. If two indexes i, j are close enough and positioned in a way that t[i] and t[j] are on the same cache line, and t[i] t[j] are accessed by different threads, they will be ``falsely shared'' and cache coherency activity will occur, which impacts performance considerably. 

\exercise{4}

\begin{lstlisting}
i = 0; j = 0; k = 0; 
while (i<n&&j<m) {
   C[k++] = (A[i]<=B[j]) ? A[i++] : B[j++]; 
}
while (i<n) C[k++] = A[i++]; 
while (j<m) C[k++] = B[j++];
\end{lstlisting}

\answer{1}
Assuming i, j, k are in registers. \\
Example for a best case input for A and B:
A = [0,0,0,0 ...] 
B = [1,1,1,1 ...] \\
This way A[0] will be fetched with a cache miss, then B[0] will be fetched with a cache miss (stored in a register afterwards), overwriting the cache line with A[0] stored. Then A[0] will be inserted at C[0].
Then A[1] will be fetched, resulting in a cache miss, however the rest of A will be fetched and inserted into C, resulting in a cache miss every 16 elements. \\
Then all of B will be inserted into C, also resulting in a cache miss every 16 elements. \\
Total cache misses: 2 + 2n/16 = 2 + n/8, \textbf{cache miss rate:} 1/16

\answer{2}
Example for a worst case input for A and B:
A = [1,2,3,4 ...] 
B = [1,2,3,4 ...] \\
If A and B contain the same values that are also strictly increasing, alternating at every iteration first A[i] <= B[j] will hold, and in the next A[i] > B[j]. At every iteration a new element has to be fetched from memory, but a miss will always occur, as the cache line will contain elements from A[i] when an element from B[j] is needed, and vice versa.  
Total cache misses: 2n, \textbf{cache miss rate:} 1.0

\exercise{5}

\answer{1}
A parallel section is opened, with to omp for constructs. In the first for-construct every element in \emph{A} is ranked in the array \emph{B} and inserted accordingly in \emph{C}. It is marked as no wait, as there are no data conflicts. In the second for-consrtuct every element in \emph{B} is ranked in the array \emph{A} and then inserted accordingly in \emph{C}. Ranking is done sequentially by binary search in O(log(n)), inserting into \emph{C} is done for n + m elements in parallel, in O(1) time. Therefore the runtime is O((n + m)/p \cdot log(n)).


\answer{2}

% TODO make look nice, use speed up

\includegraphics[width=\plotwidth, height=\plotheight]{plot_merge1_run1_1000000_2000000.png}\\
\includegraphics[width=\plotwidth, height=\plotheight]{plot_merge1_run1_10000000_20000000.png}\\
\includegraphics[width=\plotwidth, height=\plotheight]{plot_merge1_run1_100000000_200000000.png}\\



\answer{3}
To make the merge stable, nothing has to be done to preserve relative order of equal elements in A as well as B in C.
To make sure equal elements of A come before equal elements of B: \\
Rewrite rank algorithm:\\
rank{\_}for{\_}A(a, B, m) returns a j such that B[j] < a <= B[j + 1] \\
and another algorithm: \\
rank{\_}for{\_}B(b, A, n) returns an i such that A[i] <= b < A[i + 1]


\exercise{6}

\begin{lstlisting}
void merge_corank(double A[], int n, double B[], int m, double C[])
{
  int t; // number of blocks (threads)
  int i;
  
  int coj[t+1];
  int cok[t+1];
  
  for (i=0; i<t; i++) {
    corank(i*(n+m)/t,A,n,&coj[i],B,m,&cok[i]);
  }
  coj[t] = n;
  cok[t] = m;
  
  for (i=0; i<t; i++) {
    merge(&A[coj[i]],coj[i+1]-coj[i],
	  &B[cok[i]],cok[i+1]-cok[i],
	  &C[i*(n+m)/t]);
  }
}
\end{lstlisting}

\answer{1}

\answer{2}

\answer{3}

\answer{4}

\exercise{7}

\begin{lstlisting}
void merge_divconq(double A[], int n, double B[], int m, double C[])
{
  int i;

  if (n==0) { // task parallelize for large n
    for (i=0; i<m; i++) C[i] = B[i];
  } else if (m==0) { // task parallelize for large m
    for (i=0; i<n; i++) C[i] = A[i];
  } else if (n+m<CUTOFF) {
    merge(A,n,B,m,C); // sequential merge for small problems
  } else {
    int r = n/2;
    int s = rank(A[r],B,m);
    C[r+s] = A[r];
    merge_divconq(A,r,B,s,C);
    merge_divconq(&A[r+1],n-r-1,&B[s],m-s,&C[r+s+1]);
  }
}
\end{lstlisting}

\answer{1}

\answer{2}

\answer{3}

\exercise{8}

\answer{1}

\answer{2}

\answer{3}

\exercise{9}

\answer{1}

\answer{2}

\answer{3}


\end{document}
